\documentclass[11pt]{article}
% Polish
\usepackage[utf8]{inputenc}
\usepackage{polski}
\usepackage[polish]{babel}
% Graphics
\usepackage{graphicx}
% Enumerating
\usepackage{enumitem}
% Footnote bottom
\usepackage[bottom]{footmisc}
% C++
\usepackage{listings}
\usepackage{xcolor}
\lstset { %
	language=C++,
	basicstyle=\footnotesize,
	morekeywords={event_t, event_set, event_notice, event_key},
}
% Title
\title{Podstawy Sztucznej Inteligencji - Projekt \\ \large Problem Komiwojażera}
\author{A. Białobrzeski, D. Bułak, M. Waszak}
\date{\today}

\begin{document}
\maketitle

\section{Treść projektu}
Należy wybrać N miast polskich i znaleźć najkrótszy cykl łączący je wszystkie.\\
Odległości między miastami można znaleźć np. pod adresem
http://www.odleglosci.pl/tabele-odleglosci.php\\
Algorytm: genetyczny lub ewolucyjny ($\mu$+$\lambda$) i ($\mu$,$\lambda$).\\Należy ustalić w przybliżeniu najlepsze parametry działania obu algorytmów.\\
Interfejs: pokazuje postęp procesu optymalizacji, podaje ostateczną kolejność miast.\\
Należy sprawdzić program dla N=10, 20, 30. 


\section{Relizacja}
Program wyświetla okno graficzne, w którym wybiera się, który algorytm ma być stosowany (($\mu$+$\lambda$) albo ($\mu$,$\lambda$)), wprowadza się parametry: $\mu$, $\lambda$ i ilość miast ($n$). Po kliknięciu przycisku \i{Start} następuje wybranie $n$ miast z pliku, który zawiera nazwy miast oraz ich pozycje ($latitude$ i $longitude$), w tym momencie następuje uruchomienie algorytmu.\\
Algorytm uruchamiany jest w innym wątku niż GUI tak, aby swoim działaniem nie blokował interakcji użytkownika z oknem.\\
Działanie algorytmu opisano w następnym rozdziale.\\
Algorytm zwraca obiekt typu $Tour$, który zawiera uporządkowaną od początku do końca listę miast, które komiwojażer powinien odwiedzić.\\
Tak przygotowana lista jest wyświetlana w oknie za pomocą grafu.

\section{Algorytm}
Implementacja algorytmu znajduje się przede wszystkim w klasie TourCalculator, która to jest za niego odpowiedzialna. Po uruchomieniu metody $Run()$ tej klasy wykonane zostają następujące kroki:
\begin{enumerate}
\item Zainicjalizowana przekazanymi do metody miastami zostaje nowa $\mu$-elementowa populacja $initialPopulation$.
\item Dla liczby zadanych kroków zostają wykonane operacje reprodukcji:
	\begin{enumerate}
	\item Losowana jest z populacji początkowej $\lambda$-elementowa populacja tymczasowa $tempPopulation$
	\item Populacja tymczasowa jest \textbf{krzyżowana}, a wyniki tego krzyżowania zapisywane są w populacji zreprodukowanej $repPopulation$
	\item Populacja zreprodukowana $repPopulation$ poddawana jest \textbf{mutacji}.
	\item Nowa populacja początkowa ($\mu$-elementowa) dla kolejnego kroku algorytmu wybierana jest z populacji przeznaczonej do tego wyboru poprzez losowanie z niej $mi$ elementów. Populacja przeznaczona do tego losowania tworzona jest w zależności od wybranego algorytmu:
		\begin{enumerate}
		\item Dla $(\mu + \lambda)$ są to połączone populacje początkowa i zreprodukowana.
		\item Dla $(\mu, \lambda)$ jest to tylko populacja zreprodukowana.
		\end{enumerate}
	Co więcej, jeśli stosowana jest strategia elitarna, najlepszy osobnik z populacji początkowej jest przenoszony do kolejnej populacji początkowej (w nastepnej iteracji).
	\end{enumerate}
	\item Po zakończeniu wszystkich zadanych iteracji zwracany jest osobnik ($Tour$) o najlepszej funkcji przystosowania, czyli najkrótszej odległości łączącej wszystkie miasta.
\end{enumerate}

\subsection{Krzyżowanie}
Krzyżowanie zaimplementowane w metodzie $Crossover$ klasy $Population$ składa się z dwóch kroków:
\begin{enumerate}
\item Wybranie rodziców - metoda wybora to turniej. Polega on na stworzeniu niewielkiej populacji (przyjęto 5-elementową), do której osobniki losowane są z populacji, którą krzyżujemy. Następnie z członków turnieju wybierany jest ten o najlepszej funkcji przystosowania. Tworzymy dwa turnieje i po ich rozegraniu otrzymujemy dwoje rodziców.
\item Krzyżowanie rodziców ze sobą polega na stworzeniu nowego osobnika, którego część miast jest przeniesiona od pierwszego rodzica, a część od drugiego. Należy tutaj jednak pamiętać, że miasta nie mogą się powtarzać, więc przy przenoszeniu miast z drugiego osobnika musimy sprawdzać, czy takie miasto już w nim nie istnieje.
\end{enumerate}

\subsection{Mutacja}
Mutacja zaimplementowana w metodzie $Mutate$ klasy $Population$ polega na zmianie kolejności niektórych miast w osobnikach w sposób losowy. Czy mutacja zachodzi dla danego miasta w osobniku czy nie wynika z wylosowanej liczby, która jeśli jest większa lub równa od czynnika $MutationRate$ to mutacja zachodzi, w przeciwnym razie nie.\\
Jeśli okazuje się, że zachodzi wtedy losujemy drugie miasto w osobniku i zamieniamy miejscami z analizowanym.\\
Czynnik $MutationRate$ określa prawdopodobieństwo wystąpienia mutacji.

\end{document}