\documentclass[11pt]{article}
% Polish
\usepackage[utf8]{inputenc}
\usepackage{polski}
\usepackage[polish]{babel}
% Graphics
\usepackage{graphicx}
% Enumerating
\usepackage{enumitem}
% Footnote bottom
\usepackage[bottom]{footmisc}
% C++
\usepackage{listings}
\usepackage{xcolor}
\lstset { %
	language=C++,
	basicstyle=\footnotesize,
	morekeywords={event_t, event_set, event_notice, event_key},
}
% Title
\title{Podstawy Sztucznej Inteligencji - Projekt \\ \large Problem Komiwojażera}
\author{A. Białobrzeski, D. Bułak, M. Waszak}
\date{\today}

\begin{document}
\maketitle

\section{Treść projektu}
Należy wybrać N miast polskich i znaleźć najkrótszy cykl łączący je wszystkie.\\
Odległości między miastami można znaleźć np. pod adresem
http://www.odleglosci.pl/tabele-odleglosci.php\\
Algorytm: genetyczny lub ewolucyjny ($\mu$+$\lambda$) i ($\mu$,$\lambda$).\\Należy ustalić w przybliżeniu najlepsze parametry działania obu algorytmów.\\
Interfejs: pokazuje postęp procesu optymalizacji, podaje ostateczną kolejność miast.\\
Należy sprawdzić program dla N=10, 20, 30. 


\section{Koncepcja realizacji}


\end{document}